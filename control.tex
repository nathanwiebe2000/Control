\documentclass[superscriptaddress,aps,pra,nofootinbib,onecolumn,notitlepage,10pt]{revtex4-1}
\pdfoutput=1
\usepackage{graphicx}
\usepackage{amsthm}
\usepackage{amsmath}
\usepackage{amssymb}
\usepackage{comment}
\usepackage{placeins}
\usepackage[caption=false]{subfig}
\usepackage[colorlinks]{hyperref}
\usepackage{hypcap}
\usepackage{dsfont}
\usepackage{algorithm}
\usepackage{algorithmic}

\newcommand{\eq}[1]{Eq.~\hyperref[eq:#1]{(\ref*{eq:#1})}}
\renewcommand{\sec}[1]{\hyperref[sec:#1]{Section~\ref*{sec:#1}}}
\newcommand{\app}[1]{\hyperref[app:#1]{Appendix~\ref*{app:#1}}}
\newcommand{\tab}[1]{\hyperref[tab:#1]{Table~\ref*{tab:#1}}}
\newcommand{\fig}[1]{\hyperref[fig:#1]{Figure~\ref*{fig:#1}}}
\newcommand{\figa}[2]{\hyperref[fig:#1]{Figure~\ref*{fig:#1}#2}}
\newcommand{\figx}[2]{\hyperref[fig:#1]{Figure~\ref*{fig:#1}(#2)}}
\newcommand{\thm}[1]{\hyperref[thm:#1]{Theorem~\ref*{thm:#1}}}
\newcommand{\lem}[1]{\hyperref[lem:#1]{Lemma~\ref*{lem:#1}}}
\newcommand{\cor}[1]{\hyperref[cor:#1]{Corollary~\ref*{cor:#1}}}
\newcommand{\defn}[1]{\hyperref[def:#1]{Definition~\ref*{def:#1}}}
\newcommand{\alg}[1]{\hyperref[alg:#1]{Algorithm~\ref*{alg:#1}}}

\DeclareMathOperator*{\argmin}{\arg\!\min}
\newcommand{\tr}{\operatorname{Tr}}
\def\avg#1{\mathinner{\langle{#1}\rangle}}
\def\bra#1{\mathinner{\langle{#1}|}}
\def\ket#1{\mathinner{|{#1}\rangle}}
\newcommand{\braket}[2]{\langle #1|#2\rangle}
\newcommand{\ketbra}[2]{|#1\rangle\!\langle#2|}
\newcommand{\proj}[1]{\ket{#1}\!\!\bra{#1}}

\newcommand\R{{\mathrm {I\!R}}}
\newcommand\N{{\mathrm {I\!N}}}
\newcommand\h{{\cal H}}
\newcommand\V{{\cal V}}
\newcommand\p{{\sf p}}
\newcommand\w{{\sf w}}

%\newcommand{\qed}{$\hfill \Box$}
\newcommand\diag{{\mbox{diag\,}}}
\def\half{\tfrac{1}{2}}
\newcommand{\ignore}[1]{}

\newcommand{\ra}{{\rightarrow}}
\newcommand{\be}{\begin{equation}}
\newcommand{\ee}{\end{equation}}
\newcommand{\ba}{\begin{eqnarray}}
\newcommand{\ea}{\end{eqnarray}}

\newcommand{\nn}{\nonumber \\}
\newcommand{\kets}[1]{ |{#1} \rangle}
\newcommand{\herr}{\gamma}
\newcommand{\segs}{{ T}}
\newcommand{\mE}{{\mathbb E}}
\newcommand{\select}[1]{\textrm{select}(#1)}

\newcommand{\rad}{\xi}
\newcommand{\bs}[1]{\boldsymbol{#1}}
\newcommand{\wcomp}{\widetilde{\cal O}(N)}

\newcommand{\blu}{\color{blue}}
\newcommand{\blk}{\color{black}}
\newcommand{\red}{\color{red}}

\newtheorem{theorem}{Theorem}
\newtheorem{lemma}{Lemma}
\newtheorem{definition}{Definition}
\newtheorem{corollary}{Corollary}


% In case you want to swap \lambda and \Lambda
%\let\foo\lambda
%\let\lambda\Lambda
%\let\Lambda\foo

%\input{Qcircuit}
\begin{document}
\title{Control Capacities for Quantum Systems}
\begin{abstract}
Abstract
\end{abstract}
\maketitle
\section{Introduction}
\section{Control capacity}

\section{Quantum Control Capacities for Gaussian Quantum Mechanics}
Our first example will consider calculating a second moment control capacity for a problem in Gaussian quantum mechanics.
Such processes are vitally important in optics since Gaussian states represent the minimum uncertainty pure states
of the electromagnetic field or equivalently harmonic oscillators.  Since any quantum process that maintains the Gaussian
nature of the states can be classically simulated this means that we can analytically study such cases and properly deal
with the complexities of measurement back action.

We assume that the initial state for our control problem is, for some fixed $\sigma>0$
\begin{equation}
\ket{\psi} = \int_{-\infty}^\infty \frac{e^{-x^2/4\sigma^2}}{(2\pi \sigma^2)^{1/4}}\ket{x}\mathrm{d}x\label{eq:psi0}
\end{equation}
We further assume that the free evolultion of the system is given by the following kicked Hamiltonian,
\begin{equation}
H_{\rm free}(t) = \frac{p^2}{2m} - \sum_{n=-\infty}^\infty \delta(2t-2n) k x^2,
\end{equation}
which can be viewed as a linearization of the quantum kicked pendulum.  We choose this Hamiltonian because it leads to exponential instabilities
in the system as time increases and also because the kicked nature of the evolution makes it analytically tractable.
We assume that the correction is given by
\begin{equation}
H_{c}(t) = \sum_{n=-\infty}^\infty b_n \gamma_n p\delta(2t-[2n+1]).
\end{equation}
Here the $b_n$ are independent random variables with mean $\mu_b$ and variance $\sigma_b^2$.  The variables $\gamma_n$ will be chosen adaptively to stabilize the evolution of the state, by allowing $H_c(t)$ to translate the Gaussian back towards its rest configuration of $\bar{x} = \bra{\psi} x \ket{\psi}=0$.

Feedback requires measurement, which we model via a non-destructive Gaussian measurement.  We specifically assume that the standard deviation of the Gaussian that's projected onto by the system has standard deviation $\sigma$ but make no assumptions about the input standard deviation, $\sigma_1$, since it will deviate from $\sigma$ due to the above dynamical map.
The probability then of observing a mean value of $v_2$ given an initial Gaussian state with mean $\nu_1$ and variance $\sigma_1^2$ is
\begin{equation}
P(v_2,\sigma|v_1,\sigma_1) = \frac {{\rm e}^{-{\frac { \left( v_1-v_2 \right) ^{2}}{2\,{{
\sigma_1}}^{2}+2\,{{\sigma}}^{2}}}}}{\sqrt {{{\sigma_1}}^{2}+{{\sigma}}^{2}}
\sqrt {2\pi}}
,
\end{equation}
which is to say that we measure the state in a non-orthogonal basis that is spanned by statees of the form of~\eq{psi0}.  The value of $y$ learned from this measurement specified
the centroid of the Gaussian and allows us to discover how far the system has drifted from $\bar{x}=0$.  Also, because this form of measurement has a back action on the quantum state it captures the information disturbance feature of quantum mechanics (despite the fact that it nonetheless can be efficiently simulated by sampling).

To be clear, our objective is to stabilize $\bar{x}$ in the mean-square sense.  In order to find a linear strategy to achieve this we need to first solve the quantum dynamics of the system.
Because the map is periodic every $1$ unit of time, we only have to consider the dynamics within a single kick and feedback
\begin{equation}
\ket{\psi} \rightarrow e^{-i p^2/2m}e^{-ipb_n \gamma_n} e^{i kx^2}\ket{\psi},
\end{equation}
followed by a measurement operation.  Since $p= i\partial_x$, the action of these operators on a Gaussian state can be easily computed by Fourier transformation.  We specifically find that for
\begin{equation}
\ket{\psi(\nu)}:=\int_{-\infty}^\infty \frac{e^{-(x-\nu)^2/4\sigma^2}}{(2\pi \sigma^2)^{1/4}}\ket{x}\mathrm{d}x,
\end{equation}
that the probability density of a Gaussian with mean $\nu_1$ being measured as a Gaussian with mean $\nu_2$ is 
\begin{equation}
P(\nu_2,\sigma|\nu_1,\sigma_1) \propto e^{-\frac{2\sigma^2 m^2 \left(\nu_2 -\left (\nu_1 + \frac{(b_n\gamma_n)m + k\nu_1}{m}\right)\right)^2}{\sigma^4(4k^2+8km+8m^2)+1}}.
\end{equation}
The expected mean square error over all control settings is then for any linear strategy where $\gamma_n = \kappa \nu_1$ is
\begin{align}
&\mathbb{E}\left(\int_{-\infty}^\infty P(\nu_2,\sigma|\nu_1,\sigma_1) \nu_2^2\mathrm{d}\nu_2\right) \nonumber\\
&\qquad\qquad= \frac{1}{2}\frac{(1+(4k^2+8km+8m^2)\sigma^4+2\nu_1^2((1+(\mu_b^2+\sigma_b^2)\kappa^2+2\mu_b\kappa)m^2+(2\kappa\mu_b+2)km+k^2)\sigma^2)}{m^2\sigma^2}\label{eq:mse}
\end{align}
Calculus gives us that the optimal value of $\kappa$ for reducing the mean square error is
\begin{equation}
\kappa = \frac{-\mu_b(k+m)}{m(\mu_b^2 +\sigma_b^2)}.
\end{equation}
The initial mean-square error is $\nu_1^2 +\sigma^2$ before the kick operation is applied.  Thus by substituting the optimal value of $\kappa$ we see that the expected square error takes the form
\begin{equation}
\mathbb{E}_{b}\left(\int_{-\infty}^\infty P(\nu_2,\sigma|\nu_1,\sigma_1) \nu_2^2\mathrm{d}\nu_2\right) = C + \left(\frac{\sigma_b^2(k+m)^2}{m^2(\mu_b^2+\sigma_b^2)}\right)v_1^2,
\end{equation}
where $C$ is a constant function of $\nu_1$.
Thus it follows that if we take the expectation value over all possible paths that lead from the initial state to the state $\nu_1$ we have that
\begin{equation}
\mathbb{E} \left(\mathbb{E}_{b}\left(\int_{-\infty}^\infty P(\nu_2,\sigma|\nu_1,\sigma_1) \nu_2^2\mathrm{d}\nu_2\right)\right) = C + \left(\frac{\sigma_b^2(k+m)^2}{m^2(\mu_b^2+\sigma_b^2)}\right)\mathbb{E}(v_1^2).
\end{equation}
Thus we are stable under repeated applications of this map if $\sigma_b < \frac{m\mu_b}{\sqrt{k^2 +2km}}$.  This value makes sense because as $k/m\rightarrow \infty$ we need to take $\sigma_b \rightarrow 0$ in order to control the system.  Furthermore, we also find that if $\mu_b=0$ we cannot control the system either because we cannot be sure that we will get the direction of the correction shifts of the Gaussian correct.

Thus the second moment quantum control capacity, when restricted to linear strategies, reduces for this problem to
\begin{equation}
C_2 = \frac{1}{2}\log \left(\left[\frac{m^2}{(k+m)^2}\right]\left[1+\frac{\mu_b^2}{\sigma_b^2} \right]\right).
\end{equation}
Note the similar form to the channel capacity of an AWGN channel, which is $\frac{1}{2} \log(1+P/N)$, where $N$ is the noise level and $P$ is the transmission power.

\end{document}